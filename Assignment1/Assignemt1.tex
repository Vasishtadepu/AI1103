
\documentclass[journal,12pt,twocolumn]{IEEEtran}
\usepackage[utf8]{inputenc}
\usepackage{amsmath}
\usepackage{amssymb}
\usepackage{tcolorbox}
\title{Assignment-1}
\author{Adepu Vasisht}
\date{CS20BTECH11002}
\providecommand{\brak}[1]{\ensuremath{\left(#1\right)}}
\begin{document}

\maketitle

\section*{Problem 1.7}
It is known that 10$\%$ of certain articles manufactured are defective. What is the probability that in a random sample of 12 such articles, 9 are defective?
\section*{Solution 1.7}
The repeated selections of articles in a random sample space are Bernoulli trials. Let $X$ denote the number of times of selecting defective articles in a random sample space of 12 articles.\\

Clearly $X$, has a binomial distribution with n=12 and p=10\%=$\frac{1}{10}$=0.1
and q=1-p=1-$\frac{1}{10}$=$\frac{9}{10}$\\

$$\therefore X = B \brak{n,p}$$

The probability mass function(p.m.f) for a general binomial distribution is given by $$\Pr \brak{X=k} = {n \choose k} \times p^k\times q^{n-k}$$

In this case n=12 , p=$\frac{1}{10}$ and q=$\frac{9}{10}$, So 
$$\Pr\brak{k}={12 \choose k}\times(\frac{1}{10})^k\times(\frac{9}{10})^{12-k}$$
where $\Pr\brak{k}$ is the probability of getting $k$ number of defective articles and $k\in \{0,1,2.......12\}$\\

In this case we want $\Pr\brak{9}$ 
    
\begin{align}
  \implies\Pr\brak{9} &={12 \choose 9}\times(\frac{1}{10})^9\times(\frac{9}{10})^{3} \\\nonumber\\  &=\frac{12!}{9!3!}\times\frac{1}{10^9}\times\frac{9^3}{10^3}\\\nonumber\\
  &=\frac{12\times11\times10}{3\times2\times1}\times\frac{729}{10^{12}}\\\nonumber\\
  &=\frac{16038}{10^{11}}\\\nonumber\\\nonumber
\end{align}
 





So, the probability of there being 9 defective articles in a random sample space of 12 articles is $\Pr(9)=1.6038\times 10^{-7}$ 


\section*{Python code}
Since the probability of producing 9 defective articles is very very low we need to run a large number of simulations. But a large number of simulations takes a large amount of time which is inefficient so we need to strike a balance between time and the precision to the original value. So a simulation of $4\times10^8$ would likely yield a better results in a better time\\
The codes can be downloaded from

\begin{tcolorbox}{
Vasishtadepu/AI5002/tree/main/Assignment1\\/Codes
}
\end{tcolorbox}


\end{document}