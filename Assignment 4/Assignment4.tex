\documentclass[journal,12pt,twocolumn]{IEEEtran}
\usepackage[utf8]{inputenc}
\usepackage{amsmath}
\usepackage{amssymb}
\usepackage{tcolorbox}
\title{Assignment-4}
\author{Adepu Vasisht}
\date{CS20BTECH11002}
\providecommand{\brak}[1]{\ensuremath{\left(#1\right)}}
\begin{document}

\maketitle
\section*{GATE 2021(ST) Q.No 16}
Let $X$ be a random variable having distribution function 
\begin{equation}
\nonumber F(x)=
\begin{cases}
0, \quad x<1\\
\frac{a}{2}, \quad  1\leq x<2\\
\frac{c}{6}, \quad 2\leq x<3\\
1, \quad x\geq3
\end{cases}
\end{equation}
where $a$ and $c$ are appropriate constants. Let\\$A_n = \left[1+\frac{1}{n},3-\frac{1}{n}\right]$, $n\geq1$, and $A = \bigcup_{i=1}^{\infty}A_i$. If $\Pr\brak{X\leq1} = \frac{1}{2}$ and $E\brak{X} = \frac{5}{3}$, then $\Pr\brak{X\in A}$ equals to?

\section*{Solution}
\subsection{Finding $a$ and $c$}
Clearly we can see that $X$ is a discrete random variable since given $a$ and $c$ are constants. So $X$ can take only 3 values namely 1,2 and 3.
Since we know that 
\begin{align}
\nonumber F\brak{1} &= \frac{a}{2}\\
\nonumber \\
\nonumber \Pr\brak{X\leq1} &= \frac{1}{2}\\
\nonumber \\
\nonumber \implies\frac{a}{2} &= \frac{1}{2}
\end{align}
Now we calculate the value of $c$ using the given expectation value. Since $X$
is a discrete random variable the expectation value is given by 
$$E\brak{X} = \sum_{i=1}^{3}\Pr\brak{X=i}\cdot i$$
Since we need to find the individual probabilities we do that calculation first.
Now since $X$ is a discrete random variable we have $\Pr\brak{X=1} = \Pr\brak{X\leq1} = \frac{1}{2}$\\And also since $F\brak{x}$ is a cdf we have $F\brak{2} = \Pr\brak{X=1}+\Pr\brak{X=2}$\\
$$\implies \Pr\brak{X=2} = \frac{c}{6}-\frac{1}{2}$$
Similarly we can say that $F\brak{3} = F\brak{2} + \Pr\brak{X=3}$
$$\implies \Pr\brak{X=3} = 1 - \frac{c}{6}$$
Now since we have all the probability values we can calculate the expectation value.
\begin{align}
\nonumber E\brak{X} &= \sum_{i=1}^{3}\Pr\brak{X=i}\cdot i\\
\nonumber \\
\nonumber &= \Pr\brak{1}.1 +2\Pr\brak{2} + 3\Pr\brak{3}\\
\nonumber \\
\nonumber &= \frac{1}{2}+ 2\brak{\frac{c}{6}-\frac{1}{2}}+3\brak[1-\frac{c}{6}]\\
\nonumber \\
\nonumber &= \frac{1}{2} + 2 - \frac{c}{6}\\
\nonumber \\
\nonumber \frac{5}{3} &= \frac{5}{2} - \frac{c}{6}\\
\nonumber \\
\nonumber \implies \frac{c}{6} &= \frac{5}{2} - \frac{5}{3}\\
\nonumber \\
\nonumber \frac{c}{6} &= \frac{5}{6}\\
\nonumber \\
\nonumber \therefore c &= 5
\end{align}


\subsection{Finding the range of A}
Given that $A_n = \left[1+\frac{1}{n},3-\frac{1}{n}\right]$, $n\geq1$, and \\$A = \bigcup_{i=1}^{\infty}A_i$. We can see clearly that for any i we have $A_i \subset A_{i+1}$ and we also know the property that if $A$ is a subset of $B$ then we have $$A\cup B = B$$\\\\\\\\\\
By applying the above principal we get that $A = A_\infty$ and since when $n \rightarrow \infty \implies \frac{1}{n} \rightarrow 0$ but not equal to 0 so we have $A = A_\infty = \brak{1,3}$

\subsection{Finding the probability }
We have to find the probability of  $\Pr\brak{X \in A}$ but we know that $X$ is a discrete random variable which can only take $X =1,2 \: \text{and}\: 3 $ as values and $A = \brak{1,3} $ and since the only value which $X$ can take in $A$ is 2. We have $\Pr\brak{X\in A} = \Pr\brak{X = 2}$. And we know that  
\begin{align}
\nonumber \Pr\brak{X =2 } &= \frac{c}{6} - \frac{1}{2}\\
\nonumber \\
\nonumber \Pr\brak{X =2 } &= \frac{5}{6} - \frac{1}{2} \quad \brak{\because c =5}\\
\nonumber \\
\nonumber \implies \Pr\brak{X = 2} &= \frac{1}{3}
\end{align}
Hence \\$$\therefore \Pr\brak{X \in A} = 0.33$$
\end{document}